\section{Load Balancing for ZLog}
\label{sec:zlog}

First, we use the Mantle API to load balance components in ZLog, a distributed
shared commit log.  ZLog is an implementation of the
CORFU~\cite{balakrishnan_corfu_2012} API over Ceph; it uses the data I/O,
shared resource, file type, service metadata, and load balancing interfaces
from Malacology~\cite{sevilla:eurosys17-malacology}. One component of ZLog is
the sequencer, a network process that responds to client requests for the tail
position of the log.  Clients can append to the tail of the log as fast as the
sequencer can hand out positons. Clients could have been implemented by reading
the log and finding the last position, but this is slow. 

In practice, a storage system implementing CORFU will support a multiplicity of
independent totally-ordered logs for each application.  For this scenario
co-locating sequencers on the same physical node is not ideal but building a
load balancer that can migrate the shared resource ({\it e.g.}, the resource
that mediates access to the tail of the log) is a time-consuming, non-trivial
task.  It requires building subsystems for migrating resources, monitoring the
workloads, collecting metrics that describe the utilization on the physical
nodes, partitioning resources, maintaining cache coherence, and managing
multiple sequencers. The following experiments demonstrate the feasibility of
using the Mantle API and Malacology interfaces to alleviate load from
overloaded servers.

We also discuss latent capabilities we discovered in this process that let us
navigate different trade-offs.  We benchmark scenarios in which the storage
system manages multiple logs by using Mantle to balance sequencers across a
cluster.  Since this work focuses on Mantle atop Malacology the goal of this
section is to show that the components and subsystems that support the
Malacology interfaces provide reasonable relative performance.


