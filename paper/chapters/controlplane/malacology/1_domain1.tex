\section{Domain 1: Load Balancing for ZLog}
\label{sec:zlog}

%As we search for the best metadata balancing policies and physical design
%implementation trade-offs, we plan to rel
In practice, a storage system implementing CORFU will support a multiplicity of
independent totally-ordered logs for each application.  For this scenario
co-locating sequencers on the same physical node is not ideal but building a
load balancer that can migrate the shared resource (e.g., the resource
that mediates access to the tail of the log) is a time-consuming, non-trivial
task.  It requires building subsystems for migrating resources, monitoring the
workloads, collecting metrics that describe the utilization on the physical
nodes, partitioning resources, maintaining cache coherence, and managing
multiple sequencers. The following experiments demonstrate the feasibility of
using the mechanisms of the Malacology Load Balancing interface to inherit
these features and to alleviate load from overloaded servers.

We also discuss latent capabilities we discovered in this process that let us
navigate different trade-offs within the services themselves.  We benchmark
scenarios in which the storage system manages multiple logs by using Mantle to
balance sequencers across a cluster.  Since this work focuses on Mantle atop
Malacology the goal of this section is to show that the components and
subsystems that support the Malacology interfaces provide reasonable relative
performance, as well as to give examples of the flexibility that Malacology
provides to programmers.  This section uses a principled approach for
evaluating tunables of the interfaces and the trade-offs we discuss should be
acknowledged when building higher-level services.


