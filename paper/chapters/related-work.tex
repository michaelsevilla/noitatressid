\chapter{Background: Global Namespace Scalability}
\label{related-work}

% what is a global namespace
Namespaces resolve names to data. Traditionally, namespaces are hierarchical
and allow users to group similar data together; flat namespaces are outside the
scope of this work. Although file systems are the most popular example, other
instances include DNS, LAN network topologies, and scoping in programming
languages.  File system namespaces are popular because they fit our mental
organization as humans and part of the POSIX IO standard. The momentum of
namespaces as an abstraction and the overwhelming amounts of legacy code
written for namespaces makes the data model relatively future proof.

% single node file metadata
Whenever a file is created, modified, or deleted, the client must access the
file's metadata. File system metadata usually contains information about the
file, like size, links, access times, attributes, permissions/access control
lists (ACLs), and ownership.  Single disk POSIX IO file systems consult
metadata before seeking to the data.  First they translate the file name into
an inode. Then they use that inode to lookup metadata in an inode table, which
is at some fixed location on disk.  

% multi-node: why you should care: metadata management as a resurgence
Similarly, many distributed file systems decouple metadata from data access so
that data and metadata I/O can scale
independently~\cite{alam:pdsw2011-metadata-scaling, ghemawat:sosp2003-gfs,
hildebrand:msst2005-pnfs, weil:osdi2006-ceph, welch:fast2008-panasas,
xing:sc2009-skyfs}.  Unfortunately, decoupling metadata and data is
insufficient for scaling and many setups require customized application
solutions for dealing with metadata intensive workloads. For example, Google
has acknowledged a strain on their own metadata services because their
workloads involve many small files ({\it e.g.}, log processing) and
simultaneous clients ({\it e.g.}, MapReduce
jobs)~\cite{mckusick:acm2010-gfs-evolution}. Metadata inefficiencies have also
plagued Facebook; they migrated away from file systems for
photos~\cite{beaver:osdi2010-haystack} and aggressively concatenate and
compress small files so that their Hive queries do not overload the HDFS
namenode~\cite{thusoo:sigmod2010-facebook-infrastructure}. The elegance and
simplicity of the solutions stem from a thorough understanding of the workloads
({\it e.g.}, temperature zones at Facebook~\cite{muralidhar:osdi2014-f4}) and
are not applicable for general purpose storage systems. 

%% general optimization
%An accepted optimization for distributed file systems is to maintain
%system-specific information about accessing files.  Per-client capabilities for
%a file gives clients opportunities to optimize performance.  For example,
%CephFS clients are granted capabilities for reading, reading and updating,
%caching reads, writing, buffering writes, extending the end of a file, and
%doing lazy IO~\cite{docs:cephcaps}; GPFS allows byte range locking for writing
%different parts of a file~\cite{schmuck:fast2002-gpfs}; meanwhile, GFS does not
%let clients cache or buffer data because their applications typically
%stream~\cite{ghemawat:sosp2003-gfs}. Another system-specific piece of metadata
%common in distributed file systems is data location or layout information, such
%as striping strategies~\cite{nagle_panasas_2004, welch:fast08-panasas,
%wang:tech09-lustre, docs:cephstripe, sinnamohideen:atc2010-ursa,
%ghemawat:sosp2003-gfs}. This lets clients contact the storage servers for data
%IO after doing only one RPC to retrieve the map of data locations.  Other
%systems have similar capabilities but are not enumerated here because these
%optimizations are for data IO.  Instead we focus on similar optimizations for
%serving the metadata itself.

When developers scale file systems, the metadata service becomes the
performance critical component. The following sections present four reasons
that global namespaces fail to scale. Each section gives background on the
problem and brief view of related work. 

\section{Metadata Workloads}
% What is the metadata management problem?
Serving metadata and maintaining a POSIX namespace is challenging for
large-scale distributed file systems because accessing metadata imposes small
and frequent requests on the underlying storage
system~\cite{roselli:atec2000-FS-workloads}. As a result of this skewed
workload, serving metadata requests does not scale for sufficiently large
systems in the same way that read and write throughput
do~\cite{abad:ucc2012-mimesis, alam:pdsw2011-metadata-scaling,
weil:osdi2006-ceph}.  Despite this optimization file system metadata services
in HPC and large-scale data centers still have scalability problems because
common tasks, like checkpointing~\cite{bent_plfs_2009} or scanning the file
system~\cite{zheng:pdsw2014-batchfs}, contend for the same directories and
inodes.

% 1. This is an important/interesting problem
Although this important metadata problem was once reserved for high-performance
computing (HPC), it has recently found its way into large data centers. For
example, Google has acknowledged a strain on their own metadata services
because today's workloads often deal with many small files ({\it e.g.}, log
processing) and a large amount of simultaneous clients ({\it e.g.}, MapReduce
jobs)~\cite{mckusick:acm2010-gfs-evolution}. Metadata inefficiences have also
plagued Facebook; they migrated away from file systems for
photos~\cite{beaver:osdi2010-haystack} and aggressively concatenate and
compress many small files so their Hive queries do not impose too many small
files on the HDFS namenode~\cite{thusoo:sigmod2010-facebook-infrastructure}. 

% reaction
Developers in these domains are turning to the optimizations discussed later in
this section because their applications are well-understood ({\it e.g.},
well-defined read/write phases, synchronization only needed during certain
phases, workflows describing computation, etc.) and because these applications
wreak havoc on file systems designed for general-purpose workloads ({\it e.g.},
checkpoint-restart's N:N and N:1 create patterns~\cite{bent_plfs_2009}).

% Locality is a big part of workloads
\textbf{Spatial Locality Within Directories}: File system workloads have
locality because the namespace has semantic meaning; data stored in directories
is related and is usually accessed together.  Figure~\ref{figure:workload-tar}
shows the metadata locality when compiling the Linux source code. The ``heat"
of each directory is calculated with per-directory metadata counters, which are
tempered with an exponential decay.  The hotspots can be correlated with phases
of the job: untarring the code has high, sequential metadata load across
directories and compiling the code has hotspots in the \texttt{arch},
\texttt{kernel}, \texttt{fs}, and \texttt{mm} directories. Exploiting this
locality has positive implications for performance because it reduces the
number of requests, lowers the communication across MDS nodes, and eases memory
pressure. The Ceph~\cite{weil:osdi2006-ceph} (see also www.ceph.com) file
system (CephFS) tries to leverage this spatial, temporal, and request-type
locality in metadata intensive workloads using dynamic subtree partitioning,
but struggles to find the best degree of locality and balance.

% specific example
Locality, like accessing many files in the same directory with a ls -l, is
destroyed many MDSs because traversing the file hierarchy for each file is
slow. Despite the obvious path traversal benefits, LH fails to adapt to the
locality in a workload. It can rebalance when servers are added and removed,
but the request behavior does not dictate where and when metadata migrates.

\textbf{Temporal Locality During Create Flash Crowds}: Creates in the same
directory is problem in HPC~\cite{weil:sc2004-dyn-metadata, ren:sc2014-indexfs,
patil:fast2011-giga, zheng:pdsw2014-batchfs, sevilla:sc15-mantle}, mostly due
to checkpoint-restart~\cite{bent_plfs_2009}.  But the workload also appears in
cloud workloads~\cite{xiao:socc15-shardfs}, where systems like Hadoop use the
file abstraction to exchange work units to workers or to indicate when jobs
complete~\cite{shvachko:login2012-hdfs-scalability}. A more familiar example is
uncompressing an archive ({\it e.g.}, \texttt{tar xzf}), where the file system
services a flash crowd of creates across all directories as shown in
Figure~\ref{fig:overhead-creates}.  The workload is clients creating 100K files
in private directories in the same global namespace.

%Maintaining a file system hierarchy and file attributes is notoriously
%difficult in high-performance computing (HPC), where checkpointing behavior
%induces ``flash crowds" of clients simultaneously opening, writing, and
%destroying files in the same vicinity  ({\it e.g.}, a directory). 

\textbf{Listing Directories}:
The use case is that users often leverage the file system to check the progress
of jobs using \texttt{ls} even though this operation is notoriously
heavy-weight~\cite{carns:ipdps09-pvfs, eshel:fast10-panache} . The number of
files or size of the files is indicative of the progress. This practice is not
too different from systems that use the file system to manage the progress of
jobs; Hadoop writes to temporary files, renames them when complete, and creates
a ``DONE" file to indicate to the runtime that the task did not fail and should
not be re-scheduled on another node. In this scenario, Cudele users will not
see the progress of decoupled namespaces since their updates are not globally
visible.  To help users judge the progress of their jobs, Cudele clients have a
``namespace sync" that sends batches of updates back to the global namespace at
regular intervals.

% distributed file metadata
To combat the speed and dynamic nature of these metadata workloads, the
community has turned to metadata clusters instead of single metadata
servers~\cite{patil:fast2011-giga+,weil:osdi2006-ceph,weil:sc2004-dyn-metadata,
sinnamohideen:atc2010-ursa, xing:sc2009-skyfs}.  Applications perform better
with dedicated metadata servers~\cite{sevilla:sc15-mantle, ren:sc2014-indexfs}
but provisioning a metadata server for every client is unreasonable. This
problem is exacerbated by current hardware and software trends; for example,
HPC architectures are transitioning from complex storage stacks with burst
buffer, file system, object store, and tape tiers to more simplified stacks
with just a burst buffer and object store~\cite{bent:login16-hpc-trends}. These
types of trends put pressure on data access because more requests end up
hitting the same layer and latencies cannot be hidden while data migrates
across tiers.

\section{Global Semantics: Strong consistency}

% what is it
Access to metadata in a POSIX IO-compliant file system is strongly consistent,
so reads and writes to the same inode or directory are globally ordered.  The
benefit of strong consistency is that clients and servers have the same view of
the data, which makes state changes easier to reason about.  The cost of this
``safety" is performance.  The synchronization and serialization machinery
needed to ensure that all clients see the same state has high overhead.  To
make sure that all nodes or processes in the system are seeing the same state,
they must come to an agreement.  This limits parallelization and metadata
performance has been shown to {\it decrease} with more sockets in
Lustre~\cite{konstantinos:pdsw2014-lustre-metadata}. As a result, and because
it is simpler to implement, many distributed file systems limit the number of
threads in one metadata server~\cite{weil:osdi2006-ceph,
alam:pdsw2011-metadata-scaling, ren:sc2014-indexfs}. 

% heavy weight agreement
Coming to agreement has its own set of performance and accuracy trade-offs.
Sophisticated, standalone consensus engines like
PAXOS~\cite{lamport_parttime_1998}, Zookeeper~\cite{hunt_zookeeper_2010}, or
Chubby~\cite{burrows_chubby_2006} are common techniques for maintaining
consistent versions of state in groups of processes that may disagree, but
putting them in the data path is a large bottleneck. In fact, PAXOS is used in
Ceph and Zookeeper in Apache stacks to maintain cluster state not for doing IO.
Distributed file systems do not use this technique because of the overhead.

% leases, locks, capabilities
Many distributed file systems use state machines to guard access to file system
metadata.  These state machine are stored with traditional file system metadata
({\it e.g.}, size, owner, etc.) and they enforce the level of isolation that
clients are guaranteed while they are reading or writing a file. CephFS calls
them capabilities and they are managed by authority metadata servers~\cite{},
GPFS calls them write locks and they can be shared~\cite{}, Panasas refers to
them as locks and callbacks~\cite{}, IndexFS calls them leases and they are
dropped after a timeout~\cite{}, Lustre calls them locks and they protect
inodes, extents, and file locks with different modes of
concurrency~\cite{wang:tech09-lustre}.  Because this form of consistency has
high overhead many systems optimize performance by improving the locking
protocols, caching inodes, and relaxing consistency.

\subsection{Optimization: Lock Management}

The global view of these state machines, which we refer to as locks from now
on, are read and modified with RPCs from clients.  Single node metadata
servers have the simplest implementations but scalability limits. GFS and HDFS
have single node implementations with simple timeouts exposed as
configurations. These implementations do not scale for metadata-heavy workloads
so a natural approach to improving performance is to use a cluster to manage
locks.

% multi node metadata servers
Distributed lock management systems spread the lock request load across a
cluster of servers. One approach is to co-locate metadata servers that
handle just metadata IO with storage servers. PVFS2 lets users spin up metadata
servers on both storage and non-storage servers. The disadvantage of this
approach is resource contention and poor file system metadata locality,
respectively. Another approach is to orchestrate a dedicated metadata cluster,
that accounts for load imbalance and locality. GPFS assigns a process as the
authority of all locks which synchronizes access to metadata. Local servers
become the authority of metadata by contacting the global lock manager,
enabling optimizations like reducing RPCs. An even more decentralized approach
is to associate an authority process per inode. Servers manage parts of the
namespace and respond to client requests for locks~\cite{wang:tech09-lustre,
ren:sc2014-indexfs, weil:sc2004-dyn-metadata, welch:fast08-panasas}.  These
decentralized approaches have more complexity but are flexible enough to
service a range of workloads.

\subsection{Optimization: Cache Inodes}

The discussion above refers to server-server lock exchange, but systems can
also optimize client-server lock management. Caching inodes on both the client
and server lets clients modify metadata locally.  This reduces the number of
RPCs required to agree on the state of metadata.  For example, CephFS caches
entire inodes, Lustre caches lookups, IndexFS caches ACLs, PVFS2 maintains a
namespace cache and an attribute cache, Panasas lets clients read, cache, and
parse directories~\cite{welch:fast08-panasas}, GPFS and Panasas cache the
results of \texttt{stat()}~\cite{docs:cephcaps, schmuck:fast2002-gpfs,
wang:tech09-lustre, depardon:tech13-survey}, and GFS caches file
location/striping strategies.  Some systems, like Ursa Minor and pNFS maintain
client caches to reduce the overheads of NFS. These caches lead to performance
gains but the cache coherency mechanisms add significant complexity and
overhead for some workloads.

\subsection{Optimization: Relaxing Consistency}

% What is HPC doing?
A more disruptive technique is to relax the consistency semantics in the file
system because weaker guarantees are sufficient for applications. Following the
models pioneered by Amazon's eventual consistency and the more fine-grained
consistency models defined by Terry et al.~\cite{baseball}, these techniques
are gaining popularity because of the high overhead of maintaining strong
consistency.

Batching requests together is one form of relaxing consistency because updates
are not seen immediately. PVFS2~\cite{PVFS2} batches creates, Panasas combines
similar requests, like create/stat, together into one message, and Lustre
surfaces configurations that allow users to enable and disable batching.
Technically, batching requests is weaker than per-request strong consistency
but the technique is often acceptable in POSIX-compliant systems.

More extreme forms of batching ``decouple the namespace", where clients lock
the subtree they want exclusive access to as a way to tell the file system that
the subtree is important or may cause resource contention in the
near-future~\cite{grider:pdsw2015-marfs, zheng:pdsw2015-deltafs,
zheng:pdsw2014-batchfs, ren:sc2014-indexfs, bent:slides-twotiers}. Then the
file system can change its internal structure to optimize performance. For
example, the file system could enter a mode that prevents other clients from
interfering with the decoupled directory.  This delayed merge ({\it i.e.} a
form of eventual consistency) and relaxed durability improves performance and
scalability by avoiding the costs of RPCs, synchronization, false sharing, and
serialization.  While the performance benefits of decoupling the namespace are
obvious, applications that rely on the file system's guarantees must be
deployed on an entirely different system or re-written to coordinate strong
consistency/durability themselves.  These architectures are gaining popularity
because batch style jobs do not need the strong consistency that the file
system provides, so BatchFS~\cite{zheng:pdsw2014-batchfs} and
DeltaFS~\cite{zheng:pdsw2015-deltafs} do more client-side processing and merge
updates when the job is done.  

Even more drastic departures from POSIX IO allow writers and readers to
interfere with each other. GFS leaves the state of the file undefined rather
than consistent, forcing applications to use append rather than seek writes;
HDFS lets readers access files open for
writing~\cite{hakimzadeh:dais14-hdfs-consistency}; and CephFS offers the ``Lazy
IO" option, which lets clients buffer reads/writes even if other clients have
the file open and if the client maintains its own cache
coherency~\cite{docs:cephcaps}.  Lazy Hybrid updates and migrates metadata (for
elastic clusters) using the eventual consistency model.

\section{Global Semantics: Durability}

While durability is not specified by POSIX IO, users expect that files they
create or modify survive failures. The accepted technique for achieving
durability is append events to a journal of metadata updates.  Similar to
LFS~\cite{rosenblum:acm1992-LFS} and WAFL~\cite{hitz:wtec1994-WAFL} the
metadata journal is designed to be large (on the order of MBs) which ensures
(1) sequential writes into the storage device ({\it e.g.}, object store, local
disk, etc.) and (2) the ability for daemons to trim redundant or irrelevant
journal entries. We refer to metadata updates as a journal, but of course,
terminology varies from system to system ({\it e.g.}, operation log, event
list, etc.). 

\subsection{Optimization: Journal Format}

A big point of contention for distributed file systems is not the technique of
journaling metadata updates, rather it is the format of metadata. CephFS
employs a custom on-disk metadata format that behaves more like a ``pile
system". Alternatively, IndexFS stores its journal in LSM trees for fast
insertion and lookup. TableFS lays out the reasoning for using LSM trees: the
size of metadata (small) and a large number files lends itself well to an LSM
database and updates are are written to the local FS as large objects
(write-ahead logs, SSTables, large files). Panasas separates requests out into
separate logs to account for the semantic meaning and overhead of different
requests (``op-log" for creates and updates and ``cap-log" for capabilities).

\subsection{Optimization: Journal Safety}

We define three types of durability: global, local, and none.  Global
durability means that the client or server can fail at any time and metadata
will not be lost because it is ``safe" ({\it i.e.} striped or replicated across
a cluster). GFS achieves global durability by replicating its journal from the
master local disk to remote nodes and CephFS streams the journal into the
object store. Local durability means that metadata can be lost if the client or
server stays down after a failure. For example, BatchFS and DeltaFS have these
semantics because unwritten metadata updates are lost if the client (and/or its
disk) fails and stays down.  None means that metadata is volatile and that the
system provides no guarantees when clients or servers fail.  None is different
than local durability because regardless of the type of failure, metadata will
be lost when components die in a None configuration. Although we know of no
such system, storing the journal in a RAMDisk would be an example of a system
with a durability level of none.

Panasas assigns durability components to specific types of hardware. The
journal is stored in battery-backed NVRAM and later replicated to both remote
peers and metadata on objects. The software that writes the actual operations
behaves similar to WAFL/LFS without the cleaner. The system also stores
different kids of metadata (system vs. user, read vs. write) in different
places. For example, directories are mirrored across the cluster using RAID1.
This domain-specific mapping to hardware achieves high performance but
sacrifices software options and cost flexibility.

\section{Hierarchical Semantics}

Users identify and access file system data with a path name, which is a list of
directories completed with a file name.  File systems traverse (or resolve)
paths to check permissions and to verify that files exist. Files and
directories inherit some of the semantics from their parent directories, like
ownership group/user ID and permissions ({\it i.e.} ACLs). For some attributes,
parent directories must be updated as well, such as for access and modification
times. 

% Path traversal
To maintain these semantics, file systems must implement path traversal. Path
traversal starts at the root of the file system and checks each path component
until reaching the desired file. This type of workload has locality, since
directories at the top of a namespace are accessed more often~\cite{indexfs}
and because workloads tend to access files that are close in the namespace
spatially~\cite{weilthesis}.

\subsection{Optimization: Caching}

% address dirs at the top
To leverage the fact that directories at the top of the namespace are accessed
more often, some systems cache ``ancestor directories", {\it i.e.} metadata for
entire paths. In GIGA+, clients contact the parent and traverse down its
``partition history" to find which authority metadata server has the data but
in the follow-up work (IndexFS) improves lookups and creates by having clients
cache permissions (not attributes).  Similarily, Lazy
Hybrid~\cite{brandt:mss2003-lh} hashes the filename to locate metadata but
maintains extra per-file metadata to manage permissions.  Although these
techniques improve performance and scalability, especially for create intensive
workloads, they do not leverage the locality inherent in file system workloads.

% covering up locality drawbacks with more caches
Caching can also be used to exploit locality.  Many file systems hash the
namespace across metadata servers to distribute load evenly, but this approach
sacrifices workload locality. To compensate, these systems achieve locality by
adding a metadata cache~\cite{li:msst2006-dynamic, xing:sc2009-skyfs,
zhu:pds2008-hba}.  Caching popular inodes can help improve locality, but this
technique is limited by the size of the caches and only performs well for
temporal metadata, instead of spatial metadata locality. Xing et
al.~\cite{xing:sc2009-skyfs} reduce the storage footprint for metadata using
hierchical bloom filter arrays but this akin to putting a band-aid on the
problem rather than addressing the root cause.  Furthermore, cache coherence
requires a fair degree of sophistication, limiting its ability to dynamically
adapt to the flash crowds.

\subsection{Optimization: Metadata Distribution}

% address locality
Approaches that leverage the workload's spatial locality ({\it i.e.} requests
targeted at a subset of directories or files) focus on how metadata is
distributed across the cluster. File systems that hash their namespace spread
metadata evenly across the cluster but do not account for spatial locality.
IndexFS~\cite{patil:fast2011-giga+} tries to alleviate this problem by
distributing whole directories to different nodes.  While this is an
improvement, it does not address the fundamental data layout problem.
Table-based mapping is another metadata sharding technique, where the mapping
of path to inode is done by a centralized server or data
structure~\cite{xing:sc2009-skyfs, hildebrand:msst2005-pnfs,
thomson:fast2015-calvinfs}. These systems are static and while they may be able
to exploit locality at system install time, their ability to scale or adapt
with the workload is minimal.

% website:lustre
Another technique is to assign subtrees of the hierarchal namespace to server
nodes. Mostsystems use a static scheme to partition the namespace at setup,
which requires an administrator. Ursa Minor~\cite{sinnamohideen:atc2010-ursa}
and Farsite~\cite{doucer:osdi2006-farsite-dir} traverse the namespace to assign
related inode ranges, such as inodes in the same subtree, to servers. This
benefits performance because the metadata server nodes can act independently
without synchronizing their actions, making it easy to scale for breadth
assuming that the incoming data is evenly partitioned.  Subtree partitioning
also gets good locality, making multi-object operations and transactions more
efficient. If carefully planned, the metadata distributions can achieve both
locality and even load distribution, but their static distribution limits their
ability to adapt to hotspots/flash crowds and to maintain balance as data is
added.  Some systems, like Panasas~\cite{welch:fast2008-panasas}, allow certain
degrees of dynamicity by supporting the addition of new subtrees at runtime,
but adapting to the current workload is ignored. 

\subsection{Optimization: Load Balancing}

Research for load balancing file system metadata in response to workloads is in
its infancy.  IndexFS shards directories across the cluster of metadata server
when a file threshold is reached. This approach is relatively simple but is
only optimized for create-heavy workloads. CephFS has the ability to shard
directories or move entire directories to different metadata servers. This
approach has the potential to work for many types of workloads but is
relatively complicated.  The biggest problem is making policies for guiding
migration decisions. For example, deciding when to migrate load, where to
migrate load, and how much of the load to migrate are all policy decisions with
different trade-offs.
