%%%%%%%%%%%%%%%%%%%%%%%%%%%%%%%%%%%%%%%%%%%%%%%%%%%%%%%%%%%%%%%%%%%%%%%%%%%%%%%%
% INTRODUCTION
\chapter{Introduction}
%%%%%%%%%%%%%%%%%%%%%%%%%%%%%%%%%%%%%%%%%%%%%%%%%%%%%%%%%%%%%%%%%%%%%%%%%%%%%%%%

% Problem: large systems are difficult to manage
Systems that process and store large amounts of data (petabytes and beyond) are
difficult to manage. Computing has reached an era where the data is too large,
the software is too complicated, the hardware is too fast, and the events are
too frequent for any human to manage. These drastic changes should alter how
large systems are designed and deployed. We argue that the most elegant and
future-proof solution for improving and maintaining performance in these large
systems is to improve the communication between applications and the storage.
Our solution is  motivated by three trends that lead to more software layers:
(1) more data, (2) extreme heterogeneity, and (3) open-source software.  The
proliferation of large complex stacks composed of many layers makes this thesis
an especially timely solution.

% Timeliness: layers on layers; {big stacks, overhead, proof}
The overwhelming volume, velocity, and veracity of today's data shapes
modern software. When data grows too large, we scale to larger systems, either
by scaling out or up. Focusing on the scale-out model has given birth to
stacks, like Apache, that are used in industry, laboratories, and academia. But
the size of these stacks leads to increased complexity, as code bases are
larger and there are more layers~\cite{sevilla:eurosys17-malacology}, resulting
in reduced performance, redundant code, and longer code paths. The overhead of
these stacks are so high that many workloads can be outperformed by a single
node with less resources~\cite{sevilla:discs2013-framework,
rowstron:hotcdp2012-hadoop-vs-single-node, schwarzkopf:hotcloud2012-7-sins,
gigaspaces:whitepaper2011-su-vs-so, michael:2007pdps-scale-up-x-scale-out}.

% Timeliness: extreme hetoregeneity {memory wall, more hardware, more runtimes}
Extreme heterogeneity in both software and hardware has also lead to larger
software stacks that manage resources. Data centers are larger and have faster
devices because device and network speeds are scaling much faster than DRAM
speeds. The so-called memory wall~\cite{wulf:sigarch1995-memory-wall} pushes
resource management into software runtimes, which must now manage large numbers
of heterogeneous devices. The increasing momentum behind disaggregated
storage~\cite{klimovic:asplos2017-reflex, klimovic:eurosys16-disagg}, a model
that uses software as the control plane and reduces the CPU requirements of
devices, is result of prognoses we are heading towards data centers that need
to provision a CPU per storage device~\cite{samuels:oss16}. Regardless of where
the future leads, the scale and complexity of software will continue to scale
with the size of the architectures they manage.

% Timeliness: oss faciliates transparency {vendor lock in, efficiency, collab}
Finally, the last trend that has lead to the explosion of software is
open-source software. Open-source software is gaining traction because it helps
consumers avoid vendor lock in, it leads to more efficient implementations, and
it encourages collaboration. All these advantages are rooted in transparency,
as developers can work together to write code that manages the extreme
heterogeneity mentioned above, but it also lets developers see the source code
for the systems they use ``off-the-shelf". In short, open-source software leads
to more software because (1) code can come from different domains,
organizations, and communities, and (2) it easier to write optimizations
because functionality is fully exposed.

In light of these trends, our solution is a concept called ``programmable
storage"~\cite{sevilla:eurosys17-malacology, watkins:hot17-declstor}.
Programmable storage facilitates the re-use and extension of existing storage
abstractions provided by the underlying software stack, to enable the creation
of new services via composition. This process is faster than reducing layers
manually for new architectures with less layers~\cite{bent:login16-hpc-trends}
that may break backwards compatibility.  We add interfaces {\it into} a storage
system's internal functionality to facilitate application co-design, leading to
more efficient implementations that inherit the robustness of the underlying
system with less code duplication. This thesis uses the programmable storage
approach to embed policy engines into file system metadata substrates to
control the behavior, performance, and transparency of the entire software
stack.

\section{Contributions}

This thesis argues that the programmable storage approach is the correct model
for scaling global namespaces and designing policies effective file system
metadata management policies. We design policies for three metadata management
techniques: subtree load balancing, subtree semantics, and implied subtrees.
The first two expand on a strong foundation of related work while the third is
a novel idea.

First, we present Mantle, a programmable file system metadata load balancer.
To help decouple policy from mechanism, we introduce a programmable storage
system that lets the designer inject custom balancing logic. We show the
flexibility and transparency of this approach by replicating the strategy of a
state-of-the-art metadata balancer and conclude by comparing this strategy to
other custom balancers on the same system. We also show how the data management
language and policy engine from Mantle turns out to be an effective control
plane for managing ZLog sequencers and ParSplice caches.  

Second, we present Cudele, an API and framework for programmable consistency
and durability in a global namespace. The system lets clients specify their
consistency/durability requirements and dynamically assign them to subtrees in
the same namespace, allowing users to optimize subtrees over time and space for
different workloads. We confirm the performance benefits of techniques
presented in related work but also explore new consistency/durability metadata
designs, all integrated over the same storage system. By custom fitting a
subtree to a create- heavy application, we show 8× speedup and can scale to
2\(\times\) as many clients when compared to our baseline system.

Third, we present Tintenfisch, a programmable file system that generates
namespaces automatically. Clients and servers are injected with a function that
describes the size and structure of a subtree and metadata is lazily populated.
This reduces the number RPCs as clients/servers need not exchange messages to
maintain consistency.

These contributions have mostly been prototyped on Ceph. Mantle was merged into
Ceph and funded by the Center for Research in Open Source Software and Los
Alamos National Laboratory. Malacology and Mantle were featured in the Next
Platform magazine and the 2017 Lua Workshop. Finally, Malacology and Cudele are
the first Popper-compliant conference papers.

\section{Research questions}

Programmable storage attempts to answer answer important data management
questions such as ``how does the workload affect where/when I move resources"
and ``how does the system's parameters affect when I move things". This project
will address the following general load balancing topics:

\subsection*{Which metrics do we use?}

To properly balance load, we have to know which metrics are important and how
they represent the state of the system. This requires understanding what
metrics are available and how to collect them. The balancer also needs to know
how the metrics are related to each other.

\subsection*{How do we quantify performance?}

After isolating the important metrics, the balancer needs to understand how the
metrics affect the global performance and behavior. This requires understanding
how over-utilized resources negatively affect performance and how system events
can indicate that the system is performing optimally.

\subsection*{What do we optimize?}

Once we have the metrics and understand how they connect to performance, we
will identify which metrics to optimize for in our load balancer. To do this,
we must understand the trade-offs of optimizing each metric. Specifically, we
must quantify how optimizing for one metric affects the other metrics in the
system. This will involve solving a multi-objective optimization problem. 

\subsection*{Which heuristics are successful?}

Finally, when we know what to optimize for, we will apply different heuristics
to resolve the optimization trade-offs. This will introduce problems that
mirror other distributed systems problems, such as:

\begin{itemize}
	\item aggressive vs. hesitant migrations	
	\item centralized vs. decentralized knowledge
	\item small vs. large migration units
	\item accurate vs. fast decision-making
	\item learning vs. forgetting rate
\end{itemize}
