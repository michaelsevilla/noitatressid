\section{Tintenfisch: File System Namespace Schemas and Generators}
\label{sec:methodology}

\begin{figure}[tb]
  \centering
  For \(n\) processes on \(m\) servers:
  \begin{itemize}
    \setlength\itemsep{-0.5em}
    \item[] \texttt{\# of dirs =} \(m \times \texttt{mkdir()}\)
    \item[] \texttt{\# of file =} \(2 \times n\)
    \item[] \texttt{\# of file per dir =} \(n/m\)
  \end{itemize}
  \caption{Function generator for PLFS} \label{fig:plfs}
\end{figure}

\begin{figure}
  \footnotesize
  \begin{minted}[xleftmargin=1em]{lua}
local box require 'box2d'
for i=_x,_x+x do for j=_y,_y+y do
  if t>30 then 
    obj_list.insert(box(x,y,z))
  else 
    b0,b1,b2,=box.nsplit(4)
    obj_list.insert(b0,b1,b2)
end end end 
return obj_list
  \end{minted}
  \caption{Code generator for SIRIUS} \label{fig:sirius}
\end{figure}
\begin{figure}
  \centering
  \footnotesize
  \begin{minted}[xleftmargin=1em]{c++}
void recurseBranch(TObjArray *o){
  TIter i(o); 
  for(TBranch *b=i.Next();
      i.Next()!=0;
      b=i.Next()){
    processBranch(b);
    recurseBranch(b->GetListOfBranches());
  }
}
  \end{minted}
  \caption{Code generator for HEP} \label{fig:hep}
\end{figure}

%local box require 'box2d'
%o = {}; i = 1   -- o: object list
%for _x=x,x+size do for _y=y,y+size do 
%  if temperature>30 then
%    box0=box.nsplit(0,2,h,x,y,z,size)
%    box1=box.nsplit(1,2,h,x,y,z,size)
%    o[i]=box0(); i=i+1
%    o[i]=box1(); i=i+1
%  else o[i] = _x.._y..z.."_0"; i=i+1 end
%end end
%return o
 

%char *tn = getTreeName().c_str();
%TTree* t = (TTree*) root->Get(tn);
%TIter i(t->GetListOfBranches());
%for(TBranch *b = i.next();
%    i.Next() != 0;
%    b = (TBranch*) i.Next())
%  recurseBranch(b->GetListOfBranches());

%For three domain-specific applications and use cases, we have identified
%scalability challenges because of the size of the namespace.  Tintensfisch
%compacts metadata by defining namespace schemas and proposing namespace
%generators.  
Namespace schemas and generators help clients and servers establish an
understanding of the final file system metadata shape and size that eliminates
the metadata overheads highlighted above.

\subsection{Namespace Schemas}
\label{sec:namespace-schemas}

Namespace schemas describe the structure of the namespace. A ``balanced"
namespace means that subtree patterns (files per directory) are repeated and a
``bounded" namespace means that the range of file/directory names can be
defined {\it a-priori} (before the job has run but after reading metadata).
Traditional shared file systems are designed for general file system workloads,
like user home directories, which have an unbalanced and unbounded namespace
schema because users can create any number of files in any pattern.  PLFS has a
balanced and bounded namespace because the distribution of files per directory
is fixed (and repeated) and any subtree can be generated using the client
hostnames and the number of processes.  ROOT and SIRIUS are examples of
unbalanced and bounded namespace schemas. The file per directory shape is not
repeated (it is determined by application-specific metadata, LRH for ROOT or
variables for SIRIUS) but the range of file/directory names can be determined
before the job starts.

\subsection{Namespace Generators}
\label{sec:namespace-generators}

A namespace generator is a compact representation of a namespace that lets
clients/servers generate file system metadata. They can be used for bounded or
balanced namespace schemas.  Tintenfisch is built on
Cudele~\cite{sevilla:ipdps18-cudele} so a centralized, globally consistent
metadata service can decouple subtrees and clients can do metadata IO locally
with the consistency/durability semantics they require. This concept is similar
to LWFS~\cite{oldfield:cc06-lwfs}, which supplied a core set of functionality
and applications add additional functionality.  In Tintenfisch, namespace
generators are stored in the directory inode of the decoupled subtree using
the ``file type" interface from Malacology~\cite{sevilla:eurosys17-malacology}.
Next we discuss 3 example namespace generators.

%This is similar to push-down predicates in databases, where the application is
%providing domain-specific knowledge that the storage system knows how to
%leverage.  

% Tintenfisch relies on the user to design effective
%namespace generators that leverage domain-specific knowledge to get the highest
%performance. This programmable storage
%approach~\cite{sevilla:eurosys17-malacology} helps application developers
%tailor the storage system to the use case without having to design a new
%storage system from scratch.

\textbf{Formula Generator}: takes domain-specific information as input and
produces a list of files and directories.  For example, PLFS creates files and
directories based on the number of clients, so administrators can use the
formula in Figure~\ref{fig:plfs}, which takes as input the number of processes
and hosts in the cluster and outputs the number of directories, files, and
files per directory.  The namespace drawn in Figure~\ref{fig:tree_plfs} can be
generated using an input of 3 hosts each with 1 process. 
%With this namespace
%generator, clients can open just the container inode and then compute and
%access its contents without \texttt{lookup()} and \texttt{open()} RPCs to a
%centralized metadata service.

%For \(n\) processes on \(m\) servers:
%\begin{itemize}
%  \item[] \texttt{\# of dirs =} \(m \times \texttt{mkdir()}\)
%  \item[] \texttt{\# of file =} \(2n \times \texttt{create()} [+ 2n \times \texttt{lookup()}]\)
%  \item[] \texttt{\# of file per dir =} \(n/m\)
%\end{itemize}

\textbf{Code Generator}: gives users the flexiblity to write programs that
generate the namespace. This is useful if the logic is too complex to store as
a formula or requires external libraries to interpret metadata. For example,
SIRIUS constructs the namespace using domain-specific partitioning logic
written in Lua.  Figure~\ref{fig:sirius} shows how the namespace can be
constructed by iterating through  bounding box coordinates and checking if a
threshold temperature is eclipsed. If it is, extra names are generated using
the \texttt{box2d} package.  Although the partitioning function itself is not
realistic, it shows how code generators can accommodate namespaces that are
complex and/or require external libraries.  

\textbf{Pointer Generator}: references metadata in scalable storage and avoids
storing large amounts of metadata in inodes, which is a frowned upon in
distributed file system communities~\cite{docs:cephinternals}. This is useful
if there is no formal specification for the namespace. For example, ROOT uses
self-describing files so headers and metadata need to be read for each ROOT
file. A code generator is insufficient for generating the namespace because all
necessary metadata is in objects scattered in the object store.  A code
generator containing library code for the ROOT framework \emph{and} a pointer
generator for referencing the input to the code can be used to describe a ROOT
file system namespace.  Figure~\ref{fig:hep} shows a code generator example
where clients requesting Branches follow the pointer generator (not pictured)
to objects containing metadata. An added benefit is that Tintenfisch can lazily
construct parts of the namespace as needed, avoiding the inode problem
discussed in~\S\ref{sec:hep}.
