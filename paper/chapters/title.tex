% Title Stuff
\title{Scalable, Global Namespaces\\with Programmable Storage}
\author{Michael A. Sevilla}
\degreeyear{2018}
\degreemonth{April}
\degree{DEFENSE}
\chair{Professor Carlos Maltzahn}
\committeememberone{Professor Scott Brandt}
\committeemembertwo{Professor Peter Alvaro}
\numberofmembers{2} %% (including chair) possible: 3, 4, 5, 6
\deanlineone{Tyrus Miller}
\deanlinetwo{Vice Provost and Dean of Graduate Studies}
\deanlinethree{}
\field{Computer Science}
\campus{Santa Cruz}

% Pre-Chapter Stuff
\begin{frontmatter}
	\maketitle\copyrightpage\tableofcontents\listoffigures\listoftables
	\begin{abstract}

Global namespaces are difficult to scale. State-of-the-art file system metadata
management techniques are often integrated into `clean-slate' file systems
built from the ground up. Developers can integrate new techniques into an
existing file system (as they become available) but this approach exposes
confusing configurations to the user and is not future-proof as code needs to
be changed, which jeopardizes the system's robustness. Furthermore, both
approaches add large software layers to the stack, which degrades performance,
reduces code path efficiency, and increases complexity. The time is ripe for a
new solution that avoids layering.

This thesis shows how to make global namespaces scalable using a `programmable
storage' approach. We add interfaces to the storage system so administrators
can design file system metadata management policies without needing to
understand storage system internals. Based on these policies, the file system
can re-arrange its internal mechanisms to facilitate scalable, global
namespaces.  The dissertation discusses the design, implementation, and
evaluation of three programmable file system metadata management systems:
Mantle, a programmable file system metadata load balancer, Cudele, an API and
framework for programmable consistency and durability in a global namespace,
and Tintenfisch, a system that facilitates namespace schemas and namespace
generators for structured metadata.

Each system is implemented on CephFS, providing state-of-the-art file system
metadata management techniques to a leading open-source project. We have had
numerous collaborators and co-authors from the CephFS team and hope to build a
community around our programmable storage systems.

	\end{abstract}
	\begin{dedication}
                \vspace*{\fill}
                \noindent This thesis is dedicated to my parents Ed and Barb; we made it.
                \begin{itemize}
                  \item[] To my older sister Kimmy because she paved the way.
                  \item[] To my younger sister Maggie because I look up to her.
                  \item[] To Kelley, for believing in and cherishing our  relationship... crescit eundo.
                \end{itemize}
                \vspace*{\fill}
		%\null\vfil
		%{\large
		%\begin{center}
			%To myself,\\\vspace{12pt}
			%Perry H. Disdainful,\\\vspace{12pt}
			%the only person worthy of my company.
		%\end{center}}
                %}
		%\vfil\null
	\end{dedication}
	\begin{acknowledgements}

I thank my advisor, Carlos Maltzahn, for his support and enthusiasm. His
academic acumen made me a better researcher but his capacity for understanding
my emotions and needs helped him shape me into a better person. I also thank
Scott Brandt and Ike Nassi for sparking my interest in systems and Peter Alvaro
for ushering me to the finish line.

I would also like to thank Shel Finkelstein and Jeff LeFevre for providing the
proper motivation and context for the work, especially in relation to database
theory. Thanks to Kleoni Ioannidou for helping me in a field that she was new
to herself. She is a magnificent influence.

To Sam Fineberg and Bob Franks, I thank you for the real-world tough love and
attention to my pursuits outside of HPE. I learned so much about myself during
those three years working for you both. To Brad Settlemyer, I thank you for
believing in Mantle and its impact, even when I did not.

To my RedHat colleagues, Sage Weil, Greg Farnum, John Spray, and Patrick
Donnelly. Thank you for co-authoring papers and reading terrible drafts. I
enjoyed working with you all.

Finally, to my peers in the Systems Research Lab, Noah Watkins and Ivo Jimenez:
thank you for helping me craft this thesis; but more importantly for your
companionship. I think we did magnficient work and convinced the world that
what we are working on matters.  The teamwork manifested in top publications
that I never thought possible. I also thank Joe Buck, Dimitris Skourtis, Adam
Crume, Andrew Shewmaker, Jianshen Liu, and Reza Nasirigerdeh for their helpful
suggestions and feedback.

This work was supported by the Center for Research in Open-Source Software
(\href{www.cross.soe.ucsc.edu}{CROSS}), the Department of Energy, the National
Science Foundation, and the Los Alamos National Laboratory Los Alamos National
Laboratory is operated by Los Alamos National Security, LLC, for the National
Nuclear Security Administration of U.S. Department of Energy (Contract
DEAC52-06NA25396).

	\end{acknowledgements}
\end{frontmatter}
