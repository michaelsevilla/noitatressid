\chapter{Conclusion}
\label{chp:conclusion}

We conclude this thesis with related work and a summary of our findings.

\section{Future Work}

The design and implementations for this work are largely prototypes. The Mantle
work was merged into Ceph but there are plenty of new balancing policies we
would like to try.

\subsection{Load Balancing with Mantle}

In the Mantle chapter we only show how certain policies can improve or degrade
performance and instead focus on how the API is flexible enough to express many
strategies.  While we do not come up with a solution that is better than
state-of-the-art systems optimized for file creates ({\it e.g.}, GIGA+), we do
present a framework that allows users to study the emergent behavior of
different strategies, both in research and in the classroom. In the immediate
future, we hope to quantify the effect that policies have on performance by
running a suite of workloads over different balancers. Other future endeavors
will focus on:

\textbf{Analyzing Scalability}: our MDS cluster is small, but today's
production systems use metadata services with a small number of nodes (often
less than 5). Our balancers are robust until 20 nodes, at which point there is
increased variability in client performance for reasons that we are still
investigating. We expect to encounter problems with CephFS's architecture ({\it
e.g.}, n-way communication and memory pressure with many files), but we are
optimistic that we can try other techniques using Mantle, like GIGA+'s
autonomous load splitting, because Mantle MDS nodes independently make
decisions. 

%Mantle has already exposed 2 performance deficiencies in CephFS, so it can
%also help improve metadata protocols and system architectures.	

\textbf{Adding Complex Balancers}: the biggest reason for designing Mantle is
to be able to test more complex balancers. Mantle's ability to save state
should accommodate balancers that use request cost and statistical modeling,
control feedback loops, and machine learning.

\textbf{Analyzing Security and Safety}: in the current prototype, there is
little safety - the administrator can inject bad policies ({\it e.g.},
\texttt{while 1}) that brings the whole system down. We wrote a simulator that
checks the logic before injecting policies in the running cluster, but this
still needs to be integrated into the prototype.

%more complicated balancers
%deep dive on CephFS metadata protocols
%run policies on other components/systems

The ZLog sequencer balancing and ParSplice cache management work lays the
foundation for future work, where we will focus on formalizing a collection of
general data management policies that can be used across applications and
storage systems.  The value of such a collection eases the burden of policy
development and paves the way for automated solutions such as (1) adaptable
policies that switch to new strategies when the current strategy behaves poorly
({\it e.g.}, thrashing, making no progress, etc.), and (2) policy generation,
where new policies are constructed by examining the collection of existing
policies.  Ultimately, we hope that this automation enables control of policies
by machines instead of administrators. 

\subsection{Subtree Semantics with Cudele}
%This approach is superior to mounting different storage systems in a global
%namespace because there is no need to provision dedicated storage clusters to
%applications or move data between these systems.  
Cudele prompts many avenues for future work.  First is to co-locate HPC
workflows with real highly parallel runtimes from the cloud in the same
namespace. This setup would show how Cudele reasonably incorporates both programming
models (client-driven parallelism and user-defined workflows) at the same time
and should show large performance gains.  Second is dynamically changing
semantics of a subtree from stronger to weaker guarantees (or vice versa). This
reduces data movement across storage cluster and file system boundaries so
the results of a Hadoop job do not need to be migrated into CephFS for
other processing; instead the \oldcomment{user}\newcomment{administrator} can
change the semantics of the HDFS subtree into a CephFS subtree, which may cause
metadata/data movement to ensure strong consistency. Third is embeddable
policies, where child subtrees have specialized features but still maintain
guarantees of their parent subtrees.  For example, a RAMDisk subtree is POSIX
IO-compliant but relaxes durability constraints, so it can reside under a POSIX
IO subtree alongside a globally durable subtree.

%Finally, our prototype enables
%performance prediction because it helps administrators quantify the costs of
%each mechanism (similar to Section~\S\ref{sec:microbenchmarks}).

%Applications can use Cudele to microbenchmark their components and software,
%similar to what we did in 

%Using those
%results, they can predict how much slower their system will be if they adopt
%stronger consistency or durability.  This is a form of co-design that takes a
%``dirty-slate" approach but building just the guarantees the application needs
%from existing implementations.  
%This can also be a verification tool where performance that varies wildly from
%the predicted performance can be a red flag that something is wrong or that
%the bottleneck is not in the consistency or durability plane.



% Implied Namespaces

% Intermediate Update Bursts

% See if this helps load balancing

% executing mechanisms in parallel

\subsection{Subtree Schemas with Tintenfisch}

In a sense, the subtree schemas and generator work presented in the Tintenfisch
chapter is the most incomplete. The first step would be to design, test, and
benchmark the generators we outlined for the three use cases in high
performance computing, high energy physics, and large scale simulations.
Resource utilization profiles for network and disk bandwidth would provide the
most insights. Bolder avenues of future work include generating mappings of
names to data. For example, instead of generating file system namespaces, we
could use the same concept to change striping strategies for individual files,
altering things such as object size, location, or naming convention. Such an
implementation would help users move data amongst storage systems so they
could, for example, move data from CephFS to Lustre. Finally, we want to use
the namespace schema framework to generate application-specific metadata to
encompass domains that need metadata tagging, like for simulations that Empress
targets.

\section{Summary}

Scalable file system metadata techniques exist but they are implemented as
`clean-slate' file systems built from the ground up. Developers can integrate
techniques into their file system as they become available, but changing code
is not trivial and jeopardizes the robustness of the system. Both techniques
create large software stacks, which reduces performance, degrades efficiency,
and increases complexity. We present programmable file system interfaces,
implemented on CephFS, that let application developers specify policies that
guide file system metadata management techniques Without needing to understand
file system internals, application developers can provide domain-specific
knowledge to the storage system, which facilitates scalable, global namespaces.

Mantle is a programmable metadata load balancer for CephFS
(Chapter~\ref{chp:mantle}) but has been shown to be effective for other domains
and workloads (Chapter~\ref{chp:mantle-beyond}). For file systems we test three
policies from related work and show how the framework and specification is
effective for succinctly expressing load balancing policies over the same
storage system. For other domains, we show how the Mantle API and framework can
be used to migrate ZLog sequencers and manage ParSplice caches. The goal is to
build a general load balancing library that can be used across domains and
workloads.

Cudele is an API and framework for programmable consistency and durability
(Chatper~\ref{chp:cudele}). We show how different file system semantics,
ranging from POSIX IO to decoupled namespaces, can co-exist in the same global
namespace. We show how our prototype can realize strong/weak/invisible
consistency and global/local/no durability.

Finally, Tintenfisch shows how namespace schemas and generators can make global
namespaces even more scalable (Chapter~\ref{chp:tintenfisch}). The read
problem, namely the management, transfer, and materialization of file system
metadata, can be addressed by categorizing and generating namespaces. We
examine three different domains (high performance computing, high energy
physics, and large scale simulations) and show how the namespaces have
structure. We also propose 3 example namespace generators (formula, code, and
pointer) that fit the applications in these domains.

This thesis shows that global namespaces can be scalable if designed correctly,
using the programmable storage approach. Our evaluations range from
microbenchmarks to real-world use cases and demonstrate a future-proof
approach.
