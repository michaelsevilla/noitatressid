\chapter{Prototyping Platform: Ceph and Malacology}
\label{chp:malacology}

% Ceph background
We use CephFS to explore the metadata management problem.
Ceph~\cite{weil:osdi2006-ceph} is a distributed storage platform that stripes
and replicates data across a reliable object store called RADOS. Clients talk
directly to object storage daemons (OSDs) on individual disks by calculating
the data placement (``where" should I store my data) and location (``where" did
I store my data) using a hash-based algorithm (CRUSH). CephFS is the
POSIX-compliant file system that uses RADOS. It decouples metadata and data
access, so data IO is done directly with RADOS while all metadata operations go
to a separate metadata cluster. The MDS cluster is connected to RADOS so it can
periodically flush its state. The hierarchical namespace is kept in the
collective memory of the MDS cluster and acts as a large distributed cache.
Directories are stored in RADOS, so if the namespace is larger than memory,
parts of it can be swapped out. 


